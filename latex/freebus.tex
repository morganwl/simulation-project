\documentclass[12pt]{article}
\usepackage{amsfonts, amsmath, fancyhdr}

\pagestyle{fancy}

\topmargin = -0.8 in
\oddsidemargin = 0.0 in
\evensidemargin = 0.0 in
\textheight = 9 in
\textwidth = 6.5 in

\newtheorem{theorem}{Theorem}
\newtheorem{result}{Result}
\newtheorem{definition}{Defintion}

\title{}
\author{Morgan Wajda-Levie}

\begin{document}

\maketitle

\section{Executive Summary}

The New York City bus network is a vital component of the city's public
transit system. Any change in service and travel times is felt by over a
million daily riders, improving their quality of life and productivity, and
making public transit a more attractive alternative to environmentally
hazardous private automobiles. We asked the question, what if we could make
our buses faster by eliminating a costly loading bottleneck at every stop?
Through a simulated model of two Brooklyn bus lines, we show that removing
on-bus fare collection could save as much as \textbf{one hour} of average
travel time for each bus trip.

\section{Introduction and Background}

In 2019, before the Covid-19 pandemic, New York City Transit provided, on
average, more than 1.77 million bus rides every weekday. While ridership fell
in 2020, 2022 ridership was measured at 1,094,415 daily rides\cite{mta}.
Shortening the average bus trip by one minute would, collectively save
181,000 hours every day.

Improvements to bus infrastructure have other benefits. If bus travel becomes
faster, more citizens might choose a city bus over a taxi or personal car,
particularly when traveling between parts of the city not served by a subway
corridor. Fewer low-occupancy vehicles relieves traffic congestion, lessens
wear-and-tear on crumbling infrastructure, frees up curbside space for other
uses, and reduces harmful emissions that threaten our health and hasten
climate disaster.

While many improvements to bus service require additional infrastructure,
we propose an improvement which could be made, immediately, without any
modifications to the streets or bus fleet in any way. By eliminating fare
collection, which requires each passenger to individually purchase a ride
while a busful of passengers wait, buses can spend less time idling by the
side of the road and more time getting passengers to their destinations.

While one could attempt to estimate shorter loading times statistically, we
wanted to explore the impact that faster boarding might have on other aspects
of system performance. We performed a study on a small simulation of two bus
lines, measured over an entire day, modeled after the Brooklyn B41 and B35
lines. Considerations were made for variations in road traffic, peak and
off-peak passenger rates, and time spent waiting for buses, both at the start
of the trip and during transfers between lines.

Boarding and fare collection times were measured through a combination of
manual experiment and daily observation of the two bus lines under study, both
on the street and inside of the buses. Considerations were taken for the use
of multiple doors, different payment methods, the impact of bus capacity on
boarding times, and non-fare related events such as the deployment of
accessibility ramps.

\section{Conclusions and Recommendations}

Over the course of \textbf{some number} of trials, our experiment showed a decrease
in travel time of 28 minutes, when the mean boarding time was decreased by 20
seconds per passenger. However, with a more conservative estimate, based on
informal observations, we prefer to predict a 10 second decrease in per
passenger boarding time, saving 19 minutes of travel time per trip, a 9\%
improvement over observed travel times. With 1 million daily bus trips, this
would result in the recovery of 315,000 hours every day.

While we recommend this improvement without reservation, we had hoped to
observe more secondary improvements. Instead, we observed a strictly linear
growth in trip time from a mean of 2 seconds boarding time all the way to 22
seconds. This suggests that time spent waiting for passengers does not
contribute to later delays.

\section{Methodology}

We measured trip time from the moment the passenger arrived at the bus stop to
the moment the bus arrived at that passenger's destination. (Because changes
to disembarking time were not under review, we only credited unloading time to
passengers \emph{on} the bus, not those in the process of leaving.)

In constructing our model, we made a number of simplifying assumptions.
\begin{itemize}
    \item Passengers arrive at stops independently of each other, in
        a non-homogeneous Poisson process.
    \item Arrival rates are dependent on time of day, according to a
        deterministic function of time. In other words, peak and off-peak
        times occur at the same time every day. (Though their magnitude is
        stochastic.)
    \item Individual stops have distinct rates for loading and unloading, but
        all stops are subject to the same peak and off-peak hours.
    \item All passengers on a bus are equally likely to unload at a given
        stop, regardless of where their trip started.
    \item Bus travel times are subject to traffic conditions, but traffic
        conditions are independent of bus behavior.
    \item Buses are assumed to operate without accident or otherwise disabling
        incident.
    \item Buses are only scheduled at the beginning of their route.
\end{itemize}

For some $i$th passenger, this trip time is measured as the sum
\[
    w_i + l_i + b_i
\].

where $w, l, b$ are waiting time, boarding time, and on-bus time,
respectively.

$w_i$ is the difference of the time that the $i$th passenger arrives at stop
$s$, from the time that the bus arrives at stop $s$. The passengers arrival
time, in turn, is conditioned upon the arrival rate, $\lambda_{s,t}$, for a
stop $s$ at time $t$, or, more precisely, over the interval $t_\alpha,
t_\beta$, during which a passenger could have arrived. The bus's arrival time
is conditioned upon the time the bus left stop $s-1$ and traffic conditions
beteen $s-1$ and $s$.

The departure of a bus from stop $s-1$ is dependent on the time spent loading
and unloading passengers at stop $s-1$, and the bus's arrival time at $s-1$.

\section{General Discussion}

\end{document}
