\documentclass[12pt]{article}
\usepackage{amsfonts, amsmath, fancyhdr}
\usepackage{mathtools}


% \topmargin = -0.85 in
\oddsidemargin = 0.0 in
\evensidemargin = 0.0 in
% \textheight = 9 in
\textwidth = 6.5 in

\pagestyle{fancy}

\newtheorem{theorem}{Theorem}
\newtheorem{result}{Result}
\newtheorem{definition}{Defintion}

\newcommand{\Prob}{\mathbb P}

\title{}
\author{Morgan Wajda-Levie}

\begin{document}

\maketitle

\section{Executive Summary}

The New York City bus network is a vital component of the city's public
transit system. Any change in service and travel times is felt by over a
million daily riders, improving their quality of life and productivity, and
making public transit a more attractive alternative to environmentally
hazardous private automobiles. We asked the question, what if we could make
our buses faster by eliminating a costly loading bottleneck at every stop?
Through a simulated model of two Brooklyn bus lines, we show that removing
on-bus fare collection could save as much as \textbf{one hour} of average
travel time for each bus trip.

\section{Introduction and Background}

In 2019, before the Covid-19 pandemic, New York City Transit provided, on
average, more than 1.77 million bus rides every weekday. While ridership fell
in 2020, 2022 ridership was measured at 1,094,415 daily rides\cite{mta}.
Shortening the average bus trip by one minute would, collectively save
181,000 hours every day.

Improvements to bus infrastructure have other benefits. If bus travel becomes
faster, more citizens might choose a city bus over a taxi or personal car,
particularly when traveling between parts of the city not served by a subway
corridor. Fewer low-occupancy vehicles relieves traffic congestion, lessens
wear-and-tear on crumbling infrastructure, frees up curbside space for other
uses, and reduces harmful emissions that threaten our health and hasten
climate disaster.

While many improvements to bus service require additional infrastructure,
we propose an improvement which could be made, immediately, without any
modifications to the streets or bus fleet in any way. By eliminating fare
collection, which requires each passenger to individually purchase a ride
while a busful of passengers wait, buses can spend less time idling by the
side of the road and more time getting passengers to their destinations.

While one could attempt to estimate shorter loading times statistically, we
wanted to explore the impact that faster boarding might have on other aspects
of system performance. We performed a study on a small simulation of two bus
lines, measured over an entire day, modeled after the Brooklyn B41 and B35
lines. Considerations were made for variations in road traffic, peak and
off-peak passenger rates, and time spent waiting for buses, both at the start
of the trip and during transfers between lines.

Boarding and fare collection times were measured through a combination of
manual experiment and daily observation of the two bus lines under study, both
on the street and inside of the buses. Considerations were taken for the use
of multiple doors, different payment methods, the impact of bus capacity on
boarding times, and non-fare related events such as the deployment of
accessibility ramps.

\section{Conclusions and Recommendations}

Over the course of 3156 trials, our experiment showed a decrease
in travel time of $25 \left(\pm 1.31\right)$ minutes, when the mean boarding
time was decreased by 19 seconds per passenger. However, with a more
conservative estimate, based on informal observations, we prefer to predict a
10 second decrease in per passenger boarding time, saving
$12.7 \left(\pm0.40\right)$ minutes of travel time per trip, a 9\%
improvement over observed travel times. With 1 million daily bus trips, this
would result in the recovery of 212,000 hours every day.

While we recommend this improvement without reservation, we had hoped to
observe more secondary improvements. Instead, we observed a strictly linear
growth in trip time from a mean of 2 seconds boarding time all the way to 22
seconds. This suggests that time spent waiting for passengers does not
contribute to later delays.

\section{Methodology}

We measured trip time from the moment the passenger arrived at the bus stop to
the moment the bus arrived at that passenger's destination. (Because changes
to disembarking time were not under review, we only credited unloading time to
passengers \emph{on} the bus, not those in the process of leaving.)

\subsection{Assumptions}
In constructing our model, we made a number of simplifying assumptions.
\begin{itemize}
    \item Passengers arrive at stops independently of each other, in
        a non-homogeneous Poisson process.
    \item Arrival rates are dependent on time of day, according to a
        deterministic function of time. In other words, peak and off-peak
        times occur at the same time every day. (Though their magnitude is
        stochastic.)
    \item Individual stops have distinct rates for loading and unloading, but
        all stops are subject to the same peak and off-peak hours.
    \item All passengers on a bus are equally likely to unload at a given
        stop, regardless of where their trip started.
    \item Bus travel times are subject to traffic conditions, but traffic
        conditions are independent of bus behavior.
    \item Buses are assumed to operate without accident or otherwise disabling
        incident.
    \item Buses are only scheduled at the beginning of their route.
\end{itemize}

\subsection{Measuring the average travel time}

Letting $T_i$ be the total travel time for the $i$th passenger, and $N$ the
total number of passengers, we can find the mean travel time,
\[
    \overline{T} = \frac{1}{N}\sum_i^N T_i
\].

For some $i$th passenger, we can partition the travel time $T_i$ into three
components, waiting time, loading time, and moving time,
\[
    T_i = W_i + L_i + M_i
\]
While waiting time is dependent on the arrival time of each individual
passenger, we can observe that loading time and moving time are experienced by
every passenger on the bus or currently waiting to board the bus. Furthermore,
we can observe that the number of passengers remains constant between stops,
meaning that we can measure a single $N^B_{b,s}$ as the number of passengers
on bus $b$ between stop $s$ and $s+1$, and that this number will include all
of those passengers who boarded at stop $s$.

Because buses and trip segments and all distinct entities, we can partition,
\[
    \sum_i^N T_i = \sum_i^N W_i + \sum_b^B \sum_s^S N^B_{b,s} \left(L^B_{b,s}
        M^B_{b,s}\right)
\]

Our simulation measures, for discrete time intervals, the number of passengers
on or boarding each bus $b \in B$ at every stop $s \in S$ of its journey, the
length of those time intervals, and the \emph{sum} time spent by each
passenger waiting for bus $b$ at stop $s$.

\subsection{Generating random variables}

\subsubsection{Passenger arrivals}

Passengers arrive at stop $s \in S$ according to a Poisson process, meaning
that the total expected passengers within any given time interval is known,
but the precise arrival times of individual passengers are randomly
distributed within those time intervals. We measure these arrival rates as
$\lambda_s$.

Because our goal is to measure bus performance over an entire day, passengers
arrive at different rates, depending on the time of day. Therefore,
$\lambda_{s,(t_0,t_1)}$ must be a function of time. Using hourly passenger
counts recorded by the New York Transit Company, we found that the sum of two
\emph{beta} distributions effectively captured the distribution of passengers
over the course of a day, with one sharply concentrated distribution during
morning rush hour and one more less concentrated distribution during afternoon
rush hours.

We estimated these morning and evening rush hours according to the following
distributions,
\begin{align*}
    Y &\sim Beta(6, 14)\\
    V &\sim Beta(4, 2)\\
\end{align*}
For simplicity, we kept these distributions constant across across stops.

Letting $f_Y(t_D), f_V(t_D)$ be the probability density functions of our rush hour
distributions, where $t_D \in \mathbb R_{[0, 1]}$ is the fraction of the
elapsed day at time $t$, we can measure,
\[
    \lambda_{s,(t_0, t_1]} = \frac{1}{2}\lambda_s\int_{t_{0,D}}^{t_{1,D}} f_Y(t_D) +
    f_V(t_D) dt_V
\].
(The constant $\frac{1}{2}$ allows the sum of the two functions, when
integrated over an entire day, to equal 1.)

This integral is equal to the difference of the \emph{cumulative distribution
function}, $F(t_D)$ at times $t_1$ and $t_0$.

Finally, because various conditions can lead to particularly active or
inactive days within the city, we introduced a \emph{daily scale} variable,
generated once per simulation,
\[
    k \sim Beta(5, 5) + \frac{1}{2}
\].

Therefore, if some bus $b \in B$ arrives at stop $s \in S$ at time $t_1$, and
the last bus to load passengers from stop $s$ closed its doors at $t_0$, the
passengers waiting for bus $b$ at time $t_1$ will be,
\[
    N^W_{s,t_1} \sim Pois\left(\frac{1}{2}\lambda_s k \left(F(t_{1,D}) -
    F(t_{0,D})\right)\right)
\]

We can now generate this Poisson random variable using traditional methods.
(For this simulation, we chose to use numpy's random number generator
functions.)

\subsubsection{Passenger wait times}

We have one step before we can load passengers onto bus $b \in B$, which is to
measure the sum of the wait times for the passengers waiting at the stop at
time $t_1$.

Because passengers arrive according to an \emph{inhomogeneous} Poisson
process, their arrival times are not uniformly distributed within a given time
interval. We must instead determine the conditional probability of a
passenger's arrival at some time between the departure of the previous bus and
the arrival of the current, and generate arrival times according to that
distribution. If we can find the inverse of the cumulative density function
for this conditional probability, we can generate these times using an inverse
transform method.

We can calculate the likelihood of a passenger arriving between time 
$(t_0, t_*]$, given that the passenger arrives during the interval
$(t_0, t_1]$,
\begin{align*}
    \Prob[t_0 < t \le t_* \mid t_0 < t \le t_1] &= \frac{\Prob[t_0 < t \le t_*]}{\Prob[t_0 <
    t \le t_1]}\\
                                                &= \frac{F_Y(t_*) - F_Y(t_0) +
                                                F_V(t_*) - F_V(t_0)}
                                                {F_Y(t_1) - F_Y(t_0) +
                                                F_V(t_1) - F_V(t_0)}
\end{align*}

Ideally, we would generate some uniform variable $u \sim U(0, 1)$ and map that
probability to some value $t_*$, but, because $u$ is derived from the
combination of two non-linear functions, calculating this inverse is not easy.
We can, however, simplify our task by considering passengers arriving
according to each function separately.

If a passenger arrives according to distribution $Y$,
\begin{align*}
    \Prob[t_0 < t \le t_* \mid t_0 < t \le t_1]
    &= \frac{F_Y(t_*) - F_Y(t_0)}{F_Y(t_1) - F_Y(t_0)}\\
    \shortintertext{Letting $u \in [0, 1]$ be this probability,}
    u
    &= \frac{F_Y(t_*) - F_Y(t_0)}{F_Y(t_1) - F_Y(t_0)}\\
    \left(F_Y(t_1) - F_Y(t_0)\right)u + F_Y(t_0)
    &= F_Y(t_*)\\
    {F_Y}^{-1}\left(\left(F_Y(t_1) - F_Y(t_0)\right)u + F_Y(t_0)\right)
    &= t_*
\end{align*}



\section{General Discussion}

\end{document}
